\section{Introduction} 
This report presents some discussions to explain how to use the period enforcer algorithm proposed by Rajkumar \cite{Raj:suspension1991} to handle self-suspending real-time tasks. The report explains briefly the underlying assumption of the period enforcer algorithm, its limitations, and how it should be correctly used to handle self-suspending sporadic real-time tasks. 

Self-suspending tasks can be classified into two models: \emph{dynamic} self-suspension and \emph{segmented} (or \emph{multi-segment}) self-suspension models. 
The dynamic self-suspension sporadic task model characterizes each
task $\tau_i$ as a $4$-tuple $(C_i,S_i,T_i,D_i)$: $T_i$ denotes the minimum inter-arrival time (or period) of $\tau_i$, $D_i$ is the relative deadline,
$C_i$ denotes the upper bound on total execution time of each job of $\tau_i$,
and $S_i$ denotes the upper bound on total suspension time of each job of $\tau_i$.  In addition to the above $4$-tuple, the segmented sporadic task model further 
characterizes the computation segments and suspension intervals as an array
$(C_{i}^1,S_{i}^1,C_{i}^2,S_{i}^2,...,S_{i}^{m_i-1},C_{i}^{m_i})$, composed of $m_i$ computation segments separated by $m_i-1$ suspension intervals. 

The basic notation in the report in \cite{Raj:suspension1991} is a \emph{deferrable task}. In Section 3 in the report in \cite{Raj:suspension1991}, it is assumed that "\emph{With deferred execution, a task $\tau_i$ can execute its $C_i$ units of execution in discrete amounts $C_i^1, C_i^2$, $\cdots$ with suspension inbetween $C_i^j$ and $C_i^{j+1}$.\footnote{The notations are alerted here to be consistent.} Without any loss of generality, we shall assume that a task $\tau_i$ can defer its entire execution time but not parts of it. hat is, a task τi executes for Ci units with no suspensions once it begins execution. Any task that does suspend after it executes for a while can be considered to be two or more tasks each with its own worst- case execution time. The only difference is that if a task $\tau_i$ is split into two tasks $\tau_i'$ followed by $\tau_i''$, then $\tau_i''$ has the same deadlines as $\tau_i′$. }". 

The deferrable execution of a deferrable task imposes a scheduling penalty because one instance of a deferrable task can first defer defer its execution by some amount of time and for the next instance without any deferrable execution. The purpose of the period enforcer algorithm is to reduce such penalty. One of the main contributions in the report in \cite{Raj:suspension1991} is its Theorem 5, as follows:
\begin{quote}
{\bf Theorem 5} \cite{Raj:suspension1991}: A deferrable task that is schedulable under its worst-case conditions is also schedulable under the period enforcer algorithm.
\end{quote}


\section{Mismatche of Self-Suspending Tasks and Deferrable Tasks}

