\documentclass[12pt]{article}
\usepackage[margin=2cm,a4paper]{geometry} 

\usepackage{times}
\usepackage{graphicx}
\usepackage{amsmath}
\usepackage{amssymb}
\usepackage{amsfonts}
% \usepackage{cite}
\usepackage[sort]{natbib}
\usepackage{setspace}
\usepackage{multido}
\usepackage[justification=Centering]{subfig}
\usepackage{algorithm}
\usepackage{algorithmic}
\usepackage{tikz}
\usepackage{multirow}
\usetikzlibrary{shadows,patterns,shapes,arrows,decorations.pathmorphing,backgrounds,positioning,fit,plotmarks}


\newenvironment{response}[0]{\textcolor{red}{\emph{Response: }}\color{blue}}{\medskip}


\setlength{\parindent}{0cm}
\setlength{\parskip}{6pt}


\newcommand{\action}[1]{\textcolor{red}{\emph{\it Actions taken: }}{\color{blue}#1}\medskip}

\begin{document}

\pagestyle{empty}

\section*{Reviewer A}

\begin{itemize}
	\item p2, line 60: assumptions on control flow $\rightarrow$ assumptions on
the task control flow
	\item p2, line 60: the segmented model is that is allows for $\rightarrow$ the
segmented model is that **it** allows for
	\item p4, line112: I would say segmented self-suspending tasks rather than
multi-segment tasks to remain consistent with the naming introduced in
the system model.
	\item p7, line 212:  to enrich the period enforcer with the following
power, consider replacing power with ``property'' or
``scheduling rule''.
	\item p7, line 221: ``it is immediately obvious that'' $\rightarrow$ I would simply
remove that part of the sentence since nothing is obvious with
self-suspending tasks and particularly for the problem addressed in this
paper.

\begin{response}
Thank you for the suggestions. We have rephrased those parts accordingly.
	
\end{response}

\end{itemize}


\section*{Reviewer B}
\begin{itemize}
	\item  First, the period enforcer analysis only applies to the case where a task
defers its entire execution time but not parts of it. Tasks that suspend
after executing for a while should be considered as two or more tasks with
their own worst-case execution time. It is known that if tasks are broken up
into multiple segments with individual relative deadlines adding up to the
original deadline, the schedulability is no longer guaranteed. For example,
a task set (C,T,D), with tasks (4, 5, 5) and (2, 10, 10) is schedulable,
whereas breaking up the (2, 10, 10) into (1.5, 10, 7.5) and (0.5, 10, 2.5)
is not schedulable.
	\item Section 3 should address the above clarification and the task $\tau_2$ should
be statically broken up before applying the period enforcer. It is true that
transforming a task set by breaking up tasks into sub tasks with sub
deadlines can lead to unschedulability on an otherwise schedulable task set.


\begin{response}
We must distinguish two cases: the period enforcer \emph{runtime rules}, and the \emph{classic analysis} of the period enforcer.

Section 3 is concerned only with the runtime rules. Specifically, it shows that the runtime rules may cause otherwise schedulable sets of multi-segment self-suspending tasks to become unschedulable. Importantly, the runtime rules do not require a conversion to single-segment deferrable tasks.

The conversion to multi-segment deferrable tasks is relevant \emph{only} in the context of schedulability analysis. However, section 3 does not discuss schedulability analysis.

To elaborate a bit more, since the period enforcer \emph{analysis} only applies to the case where a task defers its entire execution time but not parts of it, we should create the corresponding deferrable task set to carry out schedulability analysis. However, it is \emph{not} necessary to create those deferrable task sets for \emph{using} the period enforcer algorithm at runtime: since the computation segments of a segmented self-suspending task are independent from each other, we can simply use the period enforcer algorithm without knowing the corresponding deferrable task set. This is actually how we were able to create the examples in Section 2 and Section 3.

%However, there is a risk that the resulting schedule may not be feasible. Therefore, one would need to verify the feasibility of the corresponding deferrable task set before applying the period enforcer algorithm. Such a step was assumed in [21], and we explain the difficulty in this paper. The period enforcement in the example in Section 3 is only used to enforce the periodic behavior of the 2nd computation segment of task $\tau_2$. We do not specifically discuss the schedulability test over there. Our intention is to show that the period enforcer algorithm is not always superior to the original fixed-priority scheduling. That is, the enforcement makes this segment miss the deadline, but, in the fixed-priority scheduling there is no deadline miss. We leave the discussions about the task transformation and the schedulability test in Section 4. To be more precise about the above statement, we have rewritten Section 4 to answer two fundamental questions: First, how can we derive the corresponding deferrable task set efficiently? Second, how should we perform the schedulability test on the corresponding deferrable task set correctly?
\end{response}

\action{Section 2 was rewritten and expanded to clearly distinguish between the runtime rule (Section 2.2) and the classic analysis (Section 2.3). The runtime rule --- Equation (1) --- is stated in terms of multi-segmented deferrable tasks. }

\action{Section 4 was completely rewritten to more clearly explain that the problem of finding a suitable corresponding task set is open, and likely difficult.}


	\item 
Section 4 should be reworded. 'precisely converting a self-suspending task
system into a corresponding single-segment deferrable task set is not an
easy problem'  $\rightarrow$  'converting a self-suspending tasks system into a
single-segment deferrable task set introduced pessimism due to period
enforcement'. 

\begin{response}
We do not fully understand this recommendation. In any case, we have
completely rewritten this section; we hope that the current revision addresses this concern as well.
\end{response}

\item 
It is unclear how the Nelissen paper helps since it does not
assume period enforcement. If each self-suspending task is broken into
deferrable sub tasks (considering all the higher priority tasks as
non-suspending), why is the Nelissen analysis needed? Also, It should be
clarified as to why the problem of precisely computing $W_k^{j}$ is useful in
the first place since period enforcement would introduce pessimism anyway
(as shown in Section 3).



\begin{response}
We need the analysis from Nelissen et al.\ to precisely break a self-suspending task into multiple deferrable subtasks so that we can calculate the worst-case release jitter of a computation segment. Section 4 explains why it is difficult to precisely calculate the release jitter of the computation segments and clearly points out the connection to Nelissen et al.'s analysis.

%There is no additional pessimism due to the period enforcer algorithm. The pessimism is due to the assumption made by the period enforcer algorithm to allow only a set of deferrable tasks (that only suspend once at beginning). 	
\end{response}

\action{In the revision, we completely rewrote Section 4 to explain the gap between the segmented self-suspending task model and the corresponding deferrable sporadic task set, and to precisely state the equivalence of the conversion to the worst-case response-time analysis of individual computation segments (i.e., Nelissen et al.'s analysis).
}

	\item 
Section 5 should also include the scenario where period enforcement is only
applicable to the normal execution segments. It seems practical to not
enforce critical sections since they are usually short in most systems i.e.
the back-to-back execution overhead from most critical sections in
real-world scenarios should be minimal. The case 3 that should be discussed
is period enforcement applied to execution segment after the critical
section, without enforcing the critical section and just including its
back-to-back execution penalty in the worst-case analysis.


\begin{response}
Thank you for the suggestion; this is indeed an interesting option to explore. However, we believe a comprehensive exploration of all possible interpretations to be out of the scope of this paper. 

Prior work has explicitly suggested to use the period enforcer to control self-suspensions caused by the MPCP and the DPCP such that there is \emph{no} scheduling penalty, which is only possible if the period enforcer is applied to critical sections. Specifically, this is assumed in Rajkumar's 1991 book [27], and also mentioned in more recent works by his group. In fact, Rajkumar writes in the original period enforcer proposal (emphasis added):

\begin{quote}
	``The scheduling penalty of deferred execution [when using synchronization protocols] can be enormous to the extent that it can make such protocols almost useless in several cases. \emph{The development of the period enforcer algorithm was primarily motivated by the need to reduce or eliminate this penalty} with a small, if not negligible, overhead. If tasks using these synchronization protocols use the period enforcer, their worst-case blocking durations would be considerably smaller and the schedulable utilization correspondingly higher.'' [26, Section 5.1]
\end{quote}


Our intention here is to document that this simply does not work. We do not claim to evaluate all possible ways in which the period enforcer might be used otherwise.
	
\end{response}

\action{We have added Section 5.4 to briefly comment on this third possible interpretation. We now also comment on distributed semaphore protocols (e.g., DPCP, DFLP). Illustrative examples are provided in Figures~7 and~8.}

\end{itemize}


\section*{Reviewer C}

\begin{itemize}
	\item   
This reviewer was not able to download a complete version of [20] including
the figures (the .ps available online is highly corrupted). Please provide a
precise link to the full version of this paper or a way to access to it. 

\begin{response}We are not able to provide a complete version of the original paper since we are not its authors. In the preparation of the paper under review, we also relied on the same online version (without figures) to study the period enforcer algorithm. While figures are indeed missing, they can be inferred from the accompanying detailed  descriptions; the rest of the paper is not corrupted.
	
\end{response}


\action{In the revision, we have expanded and improved the description of the background material in Section 2 to make our paper much more self-contained.}


\end{itemize}


\section*{Reviewer D}


\begin{itemize}
	\item   
This paper is heavily dependent on the assumptions and analysis of the
original period enforcer algorithm in [20]. Although the authors highlighted
some of the assumptions and important result (e.g., Theorem 5) from [20], it
may be difficult for readers to understand this paper without reading in
details the analysis from [20]. The reviewer had difficulty in understanding
the discussion in section 2.3 and 2.4. After the reviewer read [20], the
discussion in section 2.3 and section 2.4 becomes clearer. Important details
from [20] need to be added to this paper to make it more readable and
self-contained.

\action{We have expanded the discussion of necessary background in Section 2. Sections 2.3 and 2.4 have been edited and clarified to make the discussion self-contained.}

	\item   
There are some terms in section 2.3 and 2.4 that the authors have used
assuming that readers are already familiar with those terms. For example,
the terms “deferrable task”, “task set transformation”,
“single-segment deferrable tasks”, “non-self-suspending sporadic
task”, etc. may be formally defined before using them. For example, the
set of transformed non-self-suspending tasks subject to release jitter
(section 2.4) essentially means a set of deferrable tasks. Since Theorem 5
uses the term deferrable task, the relationship between transformed task set
and deferrable task set may be explicitly stated for better understanding. 

\action{We added Section 2.2, which reviews in detail all relevant task models  and clearly defines the terminology used in the paper.}
	



	\item   
Theorem 5 (page 5) mentions about “worst-case” condition. What does it
mean by worst-case condition for deferrable tasks? Paper [20] explains the
worst-case. The authors could explain the worst-case or restate the theorem
in a different way so that reader does not need to seek for the worst-case
in [20].   

\action{We added an explanation of the term ``worst-case conditions'' immediately after Theorem 5.}

	\item  
Is there any way to quantify (theoretically or using simulation) the
schedulability loss/gain using period enforcer algorithm for multi-segment
self-suspending tasks? The reviewer is curious to learn the impact of the
negative (theoretical) results of period enforcer algorithm on randomly
generated multi-segment self-suspending task sets.  

\begin{response}We are not able to provide a meaningful empirical characterization of the loss/gain because there is neither a precise baseline to compare against, nor a practical analysis for the period enforcer. 
\begin{enumerate}
	\item We would be curious to learn this, too, but unfortunately we are not able to convert a segmented self-suspending task set into the corresponding deferrable task set efficiently and without  information loss. In Section 4, we explain the task set transformation for a limited special case. If there are multiple self-suspending tasks, there is no existing solution for exactly constructing the deferrable task set: the original proposal [26] failed to define such a conversion, and we did not find one that is both efficient and not too pessimistic. 
	
%\item There is no clear evidence on how such schedulability tests for the corresponding deferrable task set should be designed precisely. We have demonstrated in Section 4.2 that all the deferrable tasks (even constructed from the same self-suspending task) should be considered for analyzing the interference. But, this implies that the worst-case behaviour of the last computation segment can be very pessimistic. Based on the example provided in Section 4.2, one can imagine that the calculation of the deferrable time already considers the earlier computation segments and the calculation of the worst-case interference has to account for the earlier computation segments as well (even if there is no overlap). 
%We are not sure whether this is due to the design or due to the pessimistic analysis. Please refer to Section 4.2 for this. 

\item There are also no other known methods or analyses to handle self-suspensions in the general case without incurring a potentially large degree of pessimism; we are hence not sure what to use as a fair baseline.
\end{enumerate}

	
\end{response}

\action{Section 4 has been rewritten to  more clearly explain the difficulty of analyzing self-suspensions (both with or without the period enforcer).}

\end{itemize}

\begin{itemize}
	\item 
Based on the notes of this paper, it is not clear whether one should consider period enforcer algorithm or not for self-suspending tasks.

\begin{response}
	There is currently no known schedulability analysis for the period enforcer (if any task has more than one computation segment).  Until future work finds such an analysis, the period enforcer should not be considered as a solution for handling self-suspensions.
\end{response}

\action{We have rephrased the last paragraph of the conclusion to make this point more clearly.}


	\item    
Why the dynamic self-suspension model is presented and compared with
segmented model in section 2.1? It is not clear why period enforcer
algorithm applies only to the segmented model. Is there any reference or
argument to support this claim in line 61-62?

\action{We added Section 2.2.4 to explain this limitation.}
\end{itemize}


\end{document}
