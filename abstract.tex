\begin{abstract}
The \emph{period enforcer} algorithm for self-suspending real-time tasks is a technique for suppressing the ``back-to-back'' scheduling penalty associated with deferred execution. Originally proposed in 1991, the algorithm has attracted renewed interest in recent years. This note revisits the algorithm in the light of recent developments in the analysis of self-suspending tasks, carefully re-examines and explains its underlying assumptions and limitations, and points out three observations that have not been made in the literature to date: \textbf{(i)}~period enforcement is not strictly superior (compared to the base case without enforcement) as it can cause deadline misses in self-suspending task sets that are schedulable without enforcement; \textbf{(ii)}~to match the assumptions underlying the analysis of the period enforcer, a schedulability analysis of self-suspending tasks subject to period enforcement requires a task set  transformation that, with current techniques, is subject to exponential time complexity; and \textbf{(iii)}~the period enforcer algorithm is incompatible with all existing analyses of suspension-based locking protocols, and can in fact cause ever-increasing suspension times until a deadline is missed.
\end{abstract}