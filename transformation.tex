
\section{Deriving a Corresponding Deferrable Task Set}
\label{sec:convert}

The example in Section \ref{sec:unschedulable} can be easily converted to a corresponding single-segment deferrable task set. However, in general,  \emph{precisely} converting a self-suspending task system into a corresponding single-segment  deferrable task set is not an easy problem. We demonstrate the inherent difficulty by focusing on a special case.

Suppose that the system has $k-1$ ordinary sporadic tasks and only one segmented self-suspending task $\tau_k$. Converting a computation segment into a deferrable task requires to derive the \emph{worst-case resume time of a computation segment}, denoted as $R_k^j$ for the $j^{\mathrm{th}}$ computation segment of task $\tau_k$. Suppose that the worst-case response time of the $j^{\mathrm{th}}$ computation segment of task $\tau_k$ is $W_k^j$. It is not difficult to see that $R_k^1=0$ and $R_k^j = W_k^{j-1}+S_k^{j-1}$ for $j=2,3,\ldots,m_k-1$. We therefore need to derive the worst-case response times of the computation segments of task $\tau_k$. 

Based on these considerations, it appears that, at least for the simple example,  the problem is basically identical to the worst-case response time analysis of segmented self-suspending task systems.  However, it has been recently shown by Nelissen et al.\ \cite{ecrts15nelissen} that calculating the worst-case response time in the above ``simple'' case is already a very challenging problem. In particular, Nelissen et al.\ \cite{ecrts15nelissen} identified several misconceptions in prior analyses, and after correcting those misconceptions, observed that deriving the worst-case response time of a computation segment in pseudo-polynomial time seems to be a very challenging problem. 



In the context of the period enforcer, we consequently observe that the only existing solution to derive the \emph{precise} bound $W_k^{j}$ (and hence $R_k^j$), due to Nelissen et al.\ \cite{ecrts15nelissen},  has exponential time complexity (even for the special case above). Furthermore, as demonstrated with the example shown in Figure~\ref{fig:example}, even if the conversion is done precisely, the transformed single-segment deferrable task set can admit more pessimism than the original self-suspending task set with respect to schedulability.
