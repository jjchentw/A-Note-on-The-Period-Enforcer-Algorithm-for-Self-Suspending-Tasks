
\section{Deriving a Corresponding Deferrable Task Set}
\label{sec:convert}

To apply an analysis of the period enforcer based on Theorem 5 in~\cite{Raj:suspension1991}, we first need to convert a given set of multi-segment self-suspending tasks into a corresponding set of single-segment deferrable tasks. This raises the question: how can we efficiently derive the corresponding set of single-segment deferrable tasks? 

The original period enforcer proposal~\cite{Raj:suspension1991} is silent on this issue and does not spell out a procedure for converting a multi-segmented self-suspending task to a corresponding set of single-segment deferrable tasks. However, in our opinion, performing such a transformation without introducing additional pessimism is not at all easy in the general case.

In the following, we illustrate the inherent difficulty of the problem by focusing on a special case to which we can apply a recent result of Nelissen et al. \cite{ecrts15nelissen}, which allows analyzing the worst-case response time of multi-segmented self-suspending sporadic tasks, albeit with exponential time complexity. 
Nelissen et al.'s worst-case response time analysis~\cite{ecrts15nelissen} can be applied under the following conditions:
\begin{enumerate}
	\item the task set contains only one self-suspending task, 
	\item the self-suspending task is the lowest-priority task, 
	\item the scheduling policy is preemptive fixed-priority scheduling, and 
	\item all tasks have constrained deadlines (i.e., $D_i \leq T_i$ for all $\tau_i$).
\end{enumerate}

Even for the restrictive case above, the mixed-integer linear programming (MILP) approach by Nelissen et al. \cite{ecrts15nelissen} does not provide the exact worst-case response time. 
However, if we apply the characteristics of the worst-case release pattern provided in Lemma 2 by Nelissen et al. \cite{ecrts15nelissen},  the exact worst-case response time can be calculated by exploring all  release patterns that satisfy these conditions. Detailed discussions can be found in a recent report by Chen \cite{Chen2016b}. In fact, it has recently been shown that verifying the schedulability of such a task set is coNP-hard in the strong sense even in the considered simplified case~\cite{RTSS2016-suspension}. Therefore, analyzing the exact worst-case response time of task $\tau_k$ is NP-hard in the strong sense by following the same proof by Chen \cite{RTSS2016-suspension}.


For an arbitrary number of tasks $k \geq 2$, 
suppose that the system has $k-1$ regular sporadic tasks and only one segmented self-suspending task $\tau_k$, and that all tasks have implicit deadlines (i.e., $D_i = T_i$ for all $\tau_i$). Further suppose that task $\tau_k$ has $m_k$ segments with $m_k \geq 3$.  

To  convert a computation segment of $\tau_k$ into a single-segment deferrable task, we need to derive the segment's \emph{latest-possible arrival time}, relative to the release of a job. Formally,  for the $j^{\mathrm{th}}$ computation segment of task $\tau_k$, we let $\rho_k^j$ denote its latest-possible arrival time, with the interpretation that, if a job of task $\tau_k$ arrives at time $t$, then  it is guaranteed that the $j^{\mathrm{th}}$ computation segment of this job will not arrive later than at time $t+\rho_k^j$.

How can we compute $\rho_k^j$? Suppose that the worst-case response time of the $j^{\mathrm{th}}$ computation segment of task $\tau_k$ is $W_k^j$, and recall that $S_k^{j}$ denotes the maximum self-suspension length before the $j^{\mathrm{th}}$ computation segment of $\tau_k$. Then $\rho_k^j$ can be expressed in terms of $W_k^{j-1}$:
$$
	\rho_k^j = W_k^{j-1}+S_k^{j-1},
$$
where $W_k^0 = 0$.  Therefore, if we can derive the exact segment worst-case response time $W_k^j$ for $j=1,2,\ldots,m_k-1$, we can easily compute $\rho_k^j$  for $j=1,2,\ldots,m_k$. And conversely, if we can somehow obtain $\rho_k^j$  for $j=2,\ldots,m_k$, we  can trivially infer $W_k^j$ for $j=1,2,\ldots,m_k-1$.
Based on these considerations, it appears that the transformation problem is  --- at least in the considered special case --- equivalent to the  worst-case response time analysis of a multi-segmented self-suspending task. 

However, deriving an exact bound $W_k^j$ for $j=1,2,\ldots,m_k-1$ for task $\tau_k$ is not easy: 
even for the above ``simple'' case, Nelissen et al.'s MILP solution~\cite{ecrts15nelissen} for calculating a safe upper bound of the worst-case response time requires exponential time complexity if $j \geq 2$. 
Deriving the exact $W_k^j$ can be achieved by exploring all  release patterns that satisfy the conditions listed in Lemma 2 in \cite{ecrts15nelissen}, with very high time complexity. 
Furthermore, Nelissen et al. \cite{ecrts15nelissen} identified several misconceptions in prior analyses, and after correcting those misconceptions, observed that the problem of deriving the worst-case response time of a computation segment in pseudo-polynomial time seems to be very challenging indeed.

Nelissen et al.~\cite{ecrts15nelissen} and Chen \cite{RTSS2016-suspension} did not study the period enforcer; rather, they considered unrestricted self-suspensions. However, given that the period enforcer has no effect on tasks that do not self-suspend~\cite{Raj:suspension1991}, and given that in the considered special case only the lowest-priority task self-suspends, we believe that these observations transfer to the period enforcement case.

To summarize, to analyze the period enforcer based on Theorem~5 in~\cite{Raj:suspension1991}, a procedure for transforming multi-segmented self-suspending tasks into sets of single-segment deferrable tasks is needed, but no such procedure is given in the original proposal~\cite{Raj:suspension1991}.
%
Based on the presented considerations, we conclude that filling in this missing step is non-trivial and observe that the closest known solution by Nelissen et al.~\cite{ecrts15nelissen} requires exponential time even in the greatly simplified special case of a single self-suspending task. It thus remains unclear how Theorem~5 in~\cite{Raj:suspension1991} can be used for schedulability analysis of sets of multi-segmented self-suspending tasks. 
While we did search for alternative analysis approaches that do not rely on Theorem 5, we did not find a simple or efficient schedulability test for the period enforcer without introducing substantial additional pessimism.  The problem remains open. 

Next, we take a look at the period enforcer in the context of synchronization protocols.




%%% Local Variables:
%%% mode: latex
%%% TeX-master: "LITES/LITES-Paper.tex"
%%% End:
