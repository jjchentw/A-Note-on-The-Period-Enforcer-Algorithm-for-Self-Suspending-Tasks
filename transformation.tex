
\section{Deriving a Corresponding Deferrable Task Set}
\label{sec:convert}

The example in Section~\ref{sec:unschedulable} can be easily converted to a corresponding deferrable task set, as explained at the end of Section~\ref{sec:unschedulable}.  Due to Theorem 5 in~\cite{Raj:suspension1991}, the feasibility of the schedule by the period enforcer algorithm depends on whether the corresponding deferrable task set can be feasibly scheduled. Therefore, before applying the period enforcer algorithm to handle segmented self-suspending sporadic tasks, we need to first derive the corresponding deferrable task set as precisely as possible. 


 However, in general, such a conversion is not an easy problem. We demonstrate the inherent difficulty by focusing on a special case and by applying the recent result provided by Nelissen et al. \cite{ecrts15nelissen}, which analyzed the exact worst-case response time for segmented self-suspending sporadic tasks.
Suppose that the system has $k-1$ ordinary sporadic tasks and only one segmented self-suspending task $\tau_k$ with $D_k = T_k$.  Suppose that task $\tau_k$ has $m_k$ segments with $m_k \geq 3$.  Converting a computation segment into a deferrable task requires deriving the worst-case deferrable time, denoted as $R_k^j$, for the $j^{\mathrm{th}}$ computation segment of task $\tau_k$. Formally, if a job of task $\tau_k$ arrives at time $t$, it is guaranteed that the $j^{\mathrm{th}}$ computation segment of this job will arrive no later than $t+R_k^j$. Suppose that the worst-case response time of the $j^{\mathrm{th}}$ computation segment of task $\tau_k$ is $W_k^j$. Therefore,  if we can derive the exact $W_k^j$ for $j=1,2,\ldots,m_k-1$ for task $\tau_k$ in this special case, we can clearly conclude that $R_k^1=0$ and $R_k^j = W_k^{j-1}+S_k^{j-1}$ for $j=2,3,\ldots,m_k$.


Based on these considerations, it appears that, at least for the simple example, the problem is basically identical to the worst-case response time analysis of segmented self-suspending task systems. Deriving the exact $W_k^j$ for $j=1,2,\ldots,m_k-1$ for task $\tau_k$ is not an easy problem.  The method recently provided by Nelissen et al. \cite{ecrts15nelissen} can be used for this specific case to derive the exact $W_k^j$ if we assume that task $\tau_k$ has only the first $j$ computation segments and $j-1$ self-suspension intervals.  However, Nelissen et al. \cite{ecrts15nelissen} also showed that calculating the worst-case response time in the above ``simple'' case is already a very challenging problem, in which calculating $W_k^j$ would need exponential time complexity if $j \geq 2$.
In particular, Nelissen et al. \cite{ecrts15nelissen} identified several misconceptions in prior analyses, and after correcting those misconceptions, observed that deriving the worst-case response time of a computation segment in pseudo-polynomial time seems to be a very challenging problem. 

In the context of the period enforcer, we consequently observe that the only existing solution for deriving the \emph{precise} bound $W_k^{j}$ (and hence $R_k^j$), due to Nelissen et al.\ \cite{ecrts15nelissen},  has exponential time complexity (even for the special case above).
However, finding a safe upper bound of $W_k^j$ for $j=1,2,\ldots,m_k-1$ for task $\tau_k$ can be done in pseudo-polynomial time \cite{PH:rtss98}, if over-approximations can be tolerated. 
Furthermore, as demonstrated with the example shown in Figure~\ref{fig:example}, even if the conversion is done precisely, the transformed single-segment deferrable task set can admit more pessimism than the original self-suspending task set with respect to schedulability.
