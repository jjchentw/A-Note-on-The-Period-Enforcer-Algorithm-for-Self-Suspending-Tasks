
\section{Concluding Remarks}
\label{sec:conclusion}

We have revisited the underlying assumptions and limitations of the period enforcer algorithm, which Rajkumar \cite{Raj:suspension1991} introduced to handle segmented self-suspending real-time tasks. 

One key assumption in the original proposal \cite{Raj:suspension1991} is that a deferrable task $\tau_i$ can defer its entire execution time but not parts of it. This creates some mismatches between the original self-suspending task set and the corresponding deferrable task set, which we have demonstrated with an example that shows that Theorem 5 in \cite{Raj:suspension1991} does not reflect the schedulability of the original self-suspending task system. 


The original proposal \cite{Raj:suspension1991} further left open the question of how to convert a segmented self-suspending task set to a corresponding set of deferrable tasks. Taking into account recent developments~\cite{ecrts15nelissen}, we have observed that such a transformation is non-trivial in the general case.  

Finally, we have demonstrated that substantial difficulties arise if one attempts to combine suspension-based locks with period enforcement. These difficulties stem from the fact that period enforcement can increase contention or lock-holding times, which increases the lengths of self-suspension intervals, which then in turn feeds back into the period enforcer's minimum suspension lengths. As a consequence, period enforcement invalidates all existing blocking analyses.

Nevertheless, the period enforcer algorithm \emph{per se}, and Theorem 5 in \cite{Raj:suspension1991},  could still prove to be useful for handling self-suspending tasks (that do not use suspension-based locks) if  \emph{efficient} schedulability tests or methods for constructing sets of single-segment deferrable tasks can be found. However, such tests or transformations have not yet been found  and the development of a precise and efficient schedulability test for self-suspending tasks remains an open problem.


\section*{Acknowledgements}

We thank James H.\ Anderson and Raj Rajkumar for their comments on early drafts of this paper.
This work has been supported by DFG, as part of the Collaborative
Research Center SFB876. % (\url{http://sfb876.tu-dortmund.de/}).